\documentclass{article}
% Change "article" to "report" to get rid of page number on title page
\usepackage{amsmath,amsfonts,amsthm,amssymb}
\usepackage{setspace}
\usepackage{Tabbing}
\usepackage{fancyhdr}
\usepackage{lastpage}
\usepackage{extramarks}
\usepackage{chngpage}
\usepackage{soul,color}
\usepackage{graphicx,float,wrapfig}
\usepackage{grffile}
\usepackage{listings}
\usepackage{algorithmic}
\usepackage{algorithm}
\usepackage{pgf}
\usepackage{csvsimple}
\usepackage{pgfplotstable,filecontents}
\usepackage{tikz}
\usepackage{siunitx}
\usepackage{float}
\usetikzlibrary{arrows,automata}
\usepackage[latin1]{inputenc}

% In case you need to adjust margins:
\topmargin=-0.45in      %
\evensidemargin=0in     %
\oddsidemargin=0in      %
\textwidth=6.5in        %
\textheight=9.0in       %
\headsep=0.25in         %

% Homework Specific Information
\newcommand{\hmwkTitle}{Homework 6}
\newcommand{\hmwkDueDate}{Dec 8, 2014}
\newcommand{\hmwkClass}{Applied Information Theory}
\newcommand{\hmwkAuthorName}{Jesse Lingeman}

\newcommand{\argmax}{\mathop{\arg\max}}
\newcommand{\deriv}[2]{\frac{\partial{#1}}{\partial {#2}} }
\newcommand{\dsep}{\mbox{dsep}}
\newcommand{\Pa}{\mathop{Pa}}
\newcommand{\ND}{\mbox{ND}}
\newcommand{\De}{\mbox{De}}
\newcommand{\Ch}{\mbox{Ch}}
\newcommand{\graphG}{{\mathcal{G}}}
\newcommand{\graphH}{{\mathcal{H}}}
\newcommand{\setA}{\mathcal{A}}
\newcommand{\setB}{\mathcal{B}}
\newcommand{\setS}{\mathcal{S}}
\newcommand{\setV}{\mathcal{V}}
\DeclareMathOperator*{\union}{\bigcup}
\DeclareMathOperator*{\intersection}{\bigcap}
\DeclareMathOperator*{\Val}{Val}
\newcommand{\mbf}[1]{{\mathbf{#1}}}
\newcommand{\eq}{\!=\!}
\newcommand{\cut}[1]{{}}


% Setup the header and footer
\pagestyle{fancy}                                                       %
\lhead{\hmwkAuthorName}                                                 %
\chead{\hmwkClass: \hmwkTitle}  %
\rhead{\firstxmark}                                                     %
\lfoot{\lastxmark}                                                      %
\cfoot{}                                                                %
\rfoot{Page\ \thepage\ of\ \pageref{LastPage}}                          %
\renewcommand\headrulewidth{0.4pt}                                      %
\renewcommand\footrulewidth{0.4pt}                                      %

% This is used to trace down (pin point) problems
% in latexing a document:
%\tracingall

%%%%%%%%%%%%%%%%%%%%%%%%%%%%%%%%%%%%%%%%%%%%%%%%%%%%%%%%%%%%%
% Some tools
\newcommand{\enterProblemHeader}[1]{\nobreak\extramarks{#1}{#1 continued on next page\ldots}\nobreak%
                                    \nobreak\extramarks{#1 (continued)}{#1 continued on next page\ldots}\nobreak}%
\newcommand{\exitProblemHeader}[1]{\nobreak\extramarks{#1 (continued)}{#1 continued on next page\ldots}\nobreak%
                                   \nobreak\extramarks{#1}{}\nobreak}%

\providecommand{\norm}[1]{\lVert#1\rVert}

\newlength{\labelLength}
\newcommand{\labelAnswer}[2]
  {\settowidth{\labelLength}{#1}%
   \addtolength{\labelLength}{0.25in}%
   \changetext{}{-\labelLength}{}{}{}%
   \noindent\fbox{\begin{minipage}[c]{\columnwidth}#2\end{minipage}}%
   \marginpar{\fbox{#1}}%

   % We put the blank space above in order to make sure this
   % \marginpar gets correctly placed.
   \changetext{}{+\labelLength}{}{}{}}%

\setcounter{secnumdepth}{0}
\newcommand{\homeworkProblemName}{}%
\newcounter{homeworkProblemCounter}%
\newenvironment{homeworkProblem}[1][Problem \arabic{homeworkProblemCounter}]%
  {\stepcounter{homeworkProblemCounter}%
   \renewcommand{\homeworkProblemName}{#1}%
   \section{\homeworkProblemName}%
   \enterProblemHeader{\homeworkProblemName}}%
  {\exitProblemHeader{\homeworkProblemName}}%

\newcommand{\problemAnswer}[1]
  {\noindent\fbox{\begin{minipage}[c]{\columnwidth}#1\end{minipage}}}%

\newcommand{\problemLAnswer}[1]
  {\labelAnswer{\homeworkProblemName}{#1}}

\newcommand{\homeworkSectionName}{}%
\newlength{\homeworkSectionLabelLength}{}%
\newenvironment{homeworkSection}[1]%
  {% We put this space here to make sure we're not connected to the above.
   % Otherwise the changetext can do funny things to the other margin

   \renewcommand{\homeworkSectionName}{#1}%
   \settowidth{\homeworkSectionLabelLength}{\homeworkSectionName}%
   \addtolength{\homeworkSectionLabelLength}{0.25in}%
   \changetext{}{-\homeworkSectionLabelLength}{}{}{}%
   \subsection{\homeworkSectionName}%
   \enterProblemHeader{\homeworkProblemName\ [\homeworkSectionName]}}%
  {\enterProblemHeader{\homeworkProblemName}%

   % We put the blank space above in order to make sure this margin
   % change doesn't happen too soon (otherwise \sectionAnswer's can
   % get ugly about their \marginpar placement.
   \changetext{}{+\homeworkSectionLabelLength}{}{}{}}%

\newcommand{\sectionAnswer}[1]
  {% We put this space here to make sure we're disconnected from the previous
   % passage

   \noindent\fbox{\begin{minipage}[c]{\columnwidth}#1\end{minipage}}%
   \enterProblemHeader{\homeworkProblemName}\exitProblemHeader{\homeworkProblemName}%
   \marginpar{\fbox{\homeworkSectionName}}%

   % We put the blank space above in order to make sure this
   % \marginpar gets correctly placed.
   }%

%%%%%%%%%%%%%%%%%%%%%%%%%%%%%%%%%%%%%%%%%%%%%%%%%%%%%%%%%%%%%


%%%%%%%%%%%%%%%%%%%%%%%%%%%%%%%%%%%%%%%%%%%%%%%%%%%%%%%%%%%%%
% Make title
\title{\vspace{2in}\textmd{\textbf{\hmwkClass:\
\hmwkTitle}}\\ \hline\normalsize\vspace{0.1in}\small{Due\ on\
\hmwkDueDate}\\ \hline\vspace{0.1in}\large{}\vspace{3in}}
\date{}
\author{\textbf{\hmwkAuthorName}}
%%%%%%%%%%%%%%%%%%%%%%%%%%%%%%%%%%%%%%%%%%%%%%%%%%%%%%%%%%%%%

\begin{document}
\begin{spacing}{1.1}
%\maketitle
%\newpage
% Uncomment the \tableofcontents and \newpage lines to get a Contents page
% Uncomment the \setcounter line as well if you do NOT want subsections
%       listed in Contents
%\setcounter{tocdepth}{1}
%\tableofcontents
%\newpage

% When problems are long, it may be desirable to put a \newpage or a
% \clearpage before each homeworkProblem environment

\clearpage
\begin{center}
Jesse Lingeman, Vijay Pemmaraju \\
lingeman@cs.umass.edu, vpemmara@cs.umass.edu
\end{center}
% \subsection{A Quick Note}
% 1
\begin{homeworkProblem}[0. Contributions]
Contributions were split fairly evenly: Jesse did the Huffman Code implementation, helped debug the Hamming Code implementation, and did part of the writeup.
Vijay implemented the Hamming Code, helped with debugging overall, and did part of the writeup.
\end{homeworkProblem}

\begin{homeworkProblem}[1. Table of Character Frequencies]
We stripped out all punctuation and only alphanumeric characters and whitespace characters were left in.

\begin{tabular}{lr}
\hline
 Character   &   Frequency \\
\hline
 \\r            &        7222 \\
 \\n            &        7222 \\
             &       58759 \\
 1           &          81 \\
 0           &          28 \\
 3           &          18 \\
 2           &          30 \\
 5           &          20 \\
 4           &          20 \\
 7           &          16 \\
 6           &          13 \\
 9           &          10 \\
 8           &          30 \\
 A           &         415 \\
 C           &         271 \\
 B           &         480 \\
 E           &         201 \\
 D           &         251 \\
 G           &         185 \\
 F           &         147 \\
 \end{tabular}
 \begin{tabular}{lr}
 \hline
 Character   &   Frequency \\
 \hline
 I           &        1970 \\
 H           &         850 \\
 K           &          13 \\
 J           &          36 \\
 M           &         361 \\
 L           &         210 \\
 O           &         183 \\
 N           &         225 \\
 Q           &           9 \\
 P           &         163 \\
 S           &         666 \\
 R           &         124 \\
 U           &          51 \\
 T           &         760 \\
 W           &         540 \\
 V           &          34 \\
 Y           &         234 \\
 X           &           2 \\
 a           &       20284 \\
 c           &        6239 \\
 b           &        3363 \\
 e           &       31607 \\
 d           &       10732 \\
 g           &        4575 \\
 f           &        5396 \\
 i           &       16209 \\
 h           &       16403 \\
  \end{tabular}
 \begin{tabular}{lr}
  \hline
 Character   &   Frequency \\
 \hline
 k           &        2054 \\
 j           &         209 \\
 m           &        6630 \\
 l           &        9979 \\
 o           &       20482 \\
 n           &       17185 \\
 q           &         193 \\
 p           &        4056 \\
 s           &       15409 \\
 r           &       15417 \\
 u           &        7936 \\
 t           &       23380 \\
 w           &        6188 \\
 v           &        2688 \\
 y           &        5336 \\
 x           &         401 \\
 z           &         123 \\
\hline
\end{tabular}
\end{homeworkProblem}

\begin{homeworkProblem}[3. Compression Factor]
We achieve a compression factor of 1.8 (length of ascii bitstring / length of huffman bitstring) for our character set.
\end{homeworkProblem}

\begin{homeworkProblem}[5. Hamming Expectations]
\subsection{A}
We have set $\alpha = 0.02$ so we expect $2\%$ of the bits to be erroneous.

\subsection{B}
We expect

\subsection{C}

\subsection{D}

\end{homeworkProblem}

\begin{homeworkProblem}[6. Hamming Measurements on ASCII]
\subsection{A}
There were 334324 characters in the document.

\subsection{B}
The binary length of the document was 2674592 bits.

\subsection{C}
The channel introduced 72683 total errors.

\subsection{D}
179575 codewords had 0 errors.

\subsection{E}
54989 codewords had 1 error.

\subsection{F}
7844 codewords had 2 errors.

\subsection{G}
Only 736 codewords had more than 2 errors.

\subsection{H}
There were 19385 bit errors in the decoded string.

\subsection{I}
Error correction failed for 8580 codewords.

\subsection{J}
This results in 12732 incorrect characters.

\end{homeworkProblem}

\begin{homeworkProblem}[7. Huffman results]
We found that after decompressing the Huffman coding there were 310541 character errors (92\% error)!
This happened because the Huffman code is greedily decoded and prefix coded, so if a bit gets flipped early on and is not recovered by error correction, then the rest of the document may be off.
Even a single uncorrected bit flip can completely destroy the decompression of a Huffman code.
\end{homeworkProblem}

\end{spacing}
\end{document}

%%%%%%%%%%%%%%%%%%%%%%%%%%%%%%%%%%%%%%%%%%%%%%%%%%%%%%%%%%%%%

%----------------------------------------------------------------------%
% The following is copyright and licensing information for
% redistribution of this LaTeX source code; it also includes a liability
% statement. If this source code is not being redistributed to others,
% it may be omitted. It has no effect on the function of the above code.
%----------------------------------------------------------------------%
% Copyright (c) 2007, 2008, 2009, 2010, 2011 by Theodore P. Pavlic
%
% Unless otherwise expressly stated, this work is licensed under the
% Creative Commons Attribution-Noncommercial 3.0 United States License. To
% view a copy of this license, visit
% http://creativecommons.org/licenses/by-nc/3.0/us/ or send a letter to
% Creative Commons, 171 Second Street, Suite 300, San Francisco,
% California, 94105, USA.
%
% THE SOFTWARE IS PROVIDED "AS IS", WITHOUT WARRANTY OF ANY KIND, EXPRESS
% OR IMPLIED, INCLUDING BUT NOT LIMITED TO THE WARRANTIES OF
% MERCHANTABILITY, FITNESS FOR A PARTICULAR PURPOSE AND NONINFRINGEMENT.
% IN NO EVENT SHALL THE AUTHORS OR COPYRIGHT HOLDERS BE LIABLE FOR ANY
% CLAIM, DAMAGES OR OTHER LIABILITY, WHETHER IN AN ACTION OF CONTRACT,
% TORT OR OTHERWISE, ARISING FROM, OUT OF OR IN CONNECTION WITH THE
% SOFTWARE OR THE USE OR OTHER DEALINGS IN THE SOFTWARE.
%----------------------------------------------------------------------%
